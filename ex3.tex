
\lstinputlisting{ex3.py}

In an Einstein-de Sitter Universe where the only component in the equations of state is matter with density $\Omega_m$=1, the scale factor can be found via the Friedmann Equation as

\begin{equation}
  a(t) = \left(\frac{3}{2}H_0t\right)^{2/3}
\end{equation}

Separating the temporal and spatial components in the Second O.D.E. of density perturbations $\delta = D(t)\Delta(x)$, the differential equation can be written in just the spatial component as:

\begin{equation}
  \frac{d^2D}{dt^2}+2\frac{\dot{a}}{a}\frac{dD}{dt}=\frac{3}{2}\Omega_0H_0^2\frac{1}{a^3}D
\end{equation}

This is the linearized density growth equation. Plugging in the scale factor, where the total density $\Omega_0 = \Omega_m$:

\begin{equation}
  \frac{\delta^2D}{\delta t^2}=-\frac{4}{3t}\frac{\delta D}{\delta t}+\frac{2}{3t^2}\Omega_mD
\end{equation}

Meaning this second order differential equation can be written in the form

\begin{equation}
  \frac{\delta^2D}{\delta t^2}=f\left(t,D,\frac{\delta D}{\delta t}\right)
\end{equation}

Something like Euler's method for integrating ODE's leads to a lot of error (on the order of the step size). This is due to local truncations adding up in each step of the numerical integration. A higher order method such as Runge Kutta mitigates this truncation error due to its basic estimation of the next point from a projection forward using the current slope. Whereas Runge Kutta methods combine calculations of the slope at intermediate points between step sizes in the integration of the ODE. Fourth order RK gives the most computationally cost efficient solution, and succeeds almost always.

To use RK4, a second order differential equation like this one must be rewritten as as two coupled first order differential equations. Therefore define

\begin{equation}
  \frac{\delta D}{\delta t} \equiv z
\end{equation}

This is just a first order differential equation which RK4 is very good at solving as described above. With the above definition and the simplified Einstein-de Sitter perturbation equation, this leaves the coupled set of first order differential equations:

\begin{align*}
  \frac{\delta D}{\delta t} &= g(t,D,z)\\
  \frac{\delta z}{\delta t} &= f(t,D,z)
\end{align*}


This system can be solved for using the RK4 method in order to integrate the original second order differential equation and find the spatial density growth term.

Analytically, this was the general solution:

\begin{equation}
  D(t) = c_1t^{2/3}+c_2/t
\end{equation}

Using the initial values this leaves the solutions:

\begin{itemize}
  \item Case 1: $D(t) = 3t^{2/3}$ 
  \item Case 2: $D(t) =10/t$
  \item Case 3: $D(t) =  3t^{2/3}+ 2/t$
\end{itemize} 


The first time derivative term ($\propto \(dot{a}/a)(\delta D /\delta t)$) can be interpretted as a frictional term (sometimes called ''Hubble friction''). This means that its sign determines a lot about the growth of structure spatially, in terms of its expansion and collapse. In Case 2, a ''negative frictional term'' imparts a constraint on expansion. The large scale of the structure collapses due to some gravity type effect to a high density on the time scale of tens of years. The opposite condition is represented in Case 1 where a smaller scaled initial spatial perturbation (given by $D(1)$) and a positive frictional term can overcome collapse. An inward pressure term causes the structure to violenty expand to low density. The critical case between these two is Case 3, where a medium sized perturbation with no pressure or gravitational attraction term does not expand or contract. Its density remains constant on the time scale of thousands of years.

https://www.astro.rug.nl/~weygaert/tim1publication/lss2009/lss2009.linperturb.pdf
https://math.stackexchange.com/questions/721076/help-with-using-the-runge-kutta-4th-order-method-on-a-system-of-2-first-order-od