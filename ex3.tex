The script used to generate the results is given by:

\lstinputlisting{ex3.py}

In an Einstein-de Sitter Universe where the only component in the equations of state is matter with density $\Omega_m$=1 (and so is also spatially flat), the scale factor can be found via the Friedmann Equation as 2/3 power law in time.

By separating the temporal and spatial components in the Second O.D.E. of density perturbations $\delta = D(t)\Delta(x)$, the differential equation can be written in just the spatial component as:

\begin{equation}
  \frac{d^2D}{dt^2}+2\frac{\dot{a}}{a}\frac{dD}{dt}=\frac{3}{2}\Omega_0H_0^2\frac{1}{a^3}D
\end{equation}

This is the linearized density growth equation. Plugging in the scale factor (where the total density $\Omega_0 \rightarrow \Omega_m$):

\begin{equation}
  \frac{\delta^2D}{\delta t^2}=-\frac{4}{3t}\frac{\delta D}{\delta t}+\frac{2\Omega_m}{3t^2}D
\end{equation}

Meaning this second order differential equation can be written in the form

\begin{equation}
  \frac{\delta^2D}{\delta t^2}=f\left(t,D,\frac{\delta D}{\delta t}\right)
\end{equation}

Something like Euler's method for integrating ODE's leads to a lot of error (on the order of the step size). This is due to local truncations adding up in each step of the numerical integration. In this case Euler's method would probably suffice given the smoothness of the analytical solution(s). In general, a higher order method such as Runge Kutta mitigates this truncation error by stepping to the next point based on a weighted average of slopes at the midpoint between steps. Fourth order RK gives the most computationally cost efficient solution, and succeeds almost always. That is why RK4 was chosen over simple Euler's.

To use RK4, a second order differential equation like this one must be rewritten as as two coupled first order differential equations. Therefore, define:

\begin{equation}
  \frac{\delta D}{\delta t} \equiv z
\end{equation}

This is just a first order differential equation which RK4 is very good at solving as described above. With the above definition and the simplified Einstein-de Sitter spatial density perturbation equation, this leaves the coupled set of first order differential equations:

\begin{align*}
  \frac{\delta D}{\delta t} &= g(t,D,z)\\
  \frac{\delta z}{\delta t} &= f(t,D,z)
\end{align*}


This system can be solved for using the RK4 method in order to integrate the original second order differential equation and find the spatial density growth term.

Analytically, we can find the general solution to the second order differential equation above as:

\begin{equation}
  D(t) = c_1t^{2/3}+c_2/t
\end{equation}

aSolving for the coefficients using the initial values this leaves the solutions:

\begin{itemize}
  \item Case 1: $D(t) = 3t^{2/3}$ 
  \item Case 2: $D(t) =10/t$
  \item Case 3: $D(t) =  3t^{2/3}+ 2/t$
  \end{itemize}



\begin{figure}[h!]
  \centering
  \includegraphics[width=0.9\linewidth]{./plots/3_spatialdensitygrowth.png}
  \caption{\textbf{3}: The three provided cases of spatial density growth. Case 1 and 3 are growing modes which means the spatial density is increasing. Case 2 is a decaying mode. The analytical solutions shown above have been overplotted on the RK4 solutions down in my code. We can see good agreement between the two; there are only differences on the order of $10^{-9}$.}
  \label{fig:spatialdensitygrowth}
\end{figure}


\begin{figure}[h!]
  \centering
  \includegraphics[width=0.9\linewidth]{./plots/3_spatialdensitygrowth_residuals.png}
  \caption{\textbf{3}: The residuals between the analytical solutions with solved coefficients and my RK4 solutions show agreement to $10^{-9}$.}
  \label{fig:spatialdensitygrowthresiduals}
\end{figure}

The first time derivative term ($\propto (\dot{a}/a)(\delta D /\delta t)$) can be interpretted as a frictional term (sometimes called ''Hubble friction''). This means that its sign determines the behavior of the spatial growth of structure, in terms of expansion or collapse.


\begin{figure}[h!]
  \centering
  \includegraphics[width=0.9\linewidth]{./plots/3_spatialdensitygrowth_near_t0.png}
  \caption{\textbf{3}: Again showing the spatial growth at early times.}
  \label{fig:spatialdensitygrowtht0}
\end{figure}


The plots show two important things: growing modes (Case 1 \& Case 3) and a decaying mode (Case 2). These modes are important in analyzing the formation of structure of the early universe. In this matter dominated universe, over dense regions expand less rapidly than elsewhere. Gravitational instability causes initial growth perturbations to collapse to high density and form galaxy clusters.

The linearized perturbations growth temporal term $D(t)$ characterizes how these distributions form with time. Case 2 has a continuously declining density: the contrast in density between regions diminishes with time. At late times, it can be discarded from further considerations in the evolution of structure. A negative $D'(1)$ term prevents increasing density as a pressure against gravitational collapse. The opposite condition is represented in Case 1 where a small initial perturbation given by $D(1)$ (characterized also in the Jeans Length) and a small positive pressure allow pertubations to grow and form clusters for all time. Case 3 is somewhere between these two where it initially looks like it could expand or collapse at very early times, but the perturbations lead to exponential collapse after a few tens of years; its initial density is too low to overcome gravity.

%https://www.astro.rug.nl/~weygaert/tim1publication/lss2009/lss2009.linperturb.pdf
%https://math.stackexchange.com/questions/721076/help-with-using-the-runge-kutta-4th-order-method-on-a-system-of-2-first-order-od