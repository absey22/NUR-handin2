% EXERCISE 1(a)

\begin{subsection}{1(a)}
  
\lstinputlisting{ex1a.py}

The KS pdf is usually calculated in terms of its cdf which is given by Press et al.

\begin{equation}
  1-2\sum_{j=1}^{inf}(-1)^{j-1}\text{exp}(-2j^2z^2)
\end{equation}

This converges very rapidly (as long as z>0) in about the first three terms to reach double precision accuracy.

Then the first three terms (j=1,2,3) of the complementary distribution function give a simple sum of exponential terms.

\begin{equation}
  1-2\left( \text{exp}(-2z^2) - \text{exp}(-4z^2) + \text{exp}(-18z^2)\right)
\end{equation}

\end{subsection}


% EXERCISE 1(b)

\begin{subsection}{1(b)}
  
\lstinputlisting{ex1b.py}


\end{subsection}


% EXERCISE 1(c)

\begin{subsection}{1(c)}
  
\lstinputlisting{ex1c.py}


\end{subsection}


% EXERCISE 1(d)

\begin{subsection}{1(d)}
  
\lstinputlisting{ex1d.py}


\end{subsection}

% EXERCISE 1(e)

\begin{subsection}{1(e)}
  
\lstinputlisting{ex1e.py}


\end{subsection}
