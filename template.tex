\documentclass[a4paper,10pt]{article}
\usepackage[utf8]{inputenc}
\usepackage{placeins}
\usepackage{amsmath}
\usepackage{fullpage}
\usepackage{hyperref}
\usepackage{graphicx}
\usepackage{listings}
\usepackage{color}

\definecolor{mygreen}{rgb}{0,0.6,0}
\definecolor{mygray}{rgb}{0.5,0.5,0.5}
\definecolor{mymauve}{rgb}{0.58,0,0.82}

\lstset{ 
  backgroundcolor=\color{white},   % choose the background color; you must add \usepackage{color} or \usepackage{xcolor}; should come as last argument
  basicstyle=\footnotesize,        % the size of the fonts that are used for the code
  breakatwhitespace=false,         % sets if automatic breaks should only happen at whitespace
  breaklines=true,                 % sets automatic line breaking
  captionpos=b,                    % sets the caption-position to bottom
  commentstyle=\color{mygreen},    % comment style
  deletekeywords={...},            % if you want to delete keywords from the given language
  escapeinside={\%*}{*)},          % if you want to add LaTeX within your code
  extendedchars=true,              % lets you use non-ASCII characters; for 8-bits encodings only, does not work with UTF-8
  firstnumber=1,                   % start line enumeration with line 1000
  frame=single,	                   % adds a frame around the code
  keepspaces=true,                 % keeps spaces in text, useful for keeping indentation of code (possibly needs columns=flexible)
  keywordstyle=\color{blue},       % keyword style
  language=Python,                 % the language of the code
  morekeywords={*,...},            % if you want to add more keywords to the set
  numbers=left,                    % where to put the line-numbers; possible values are (none, left, right)
  numbersep=5pt,                   % how far the line-numbers are from the code
  numberstyle=\tiny\color{mygray}, % the style that is used for the line-numbers
  rulecolor=\color{black},         % if not set, the frame-color may be changed on line-breaks within not-black text (e.g. comments (green here))
  showspaces=false,                % show spaces everywhere adding particular underscores; it overrides 'showstringspaces'
  showstringspaces=false,          % underline spaces within strings only
  showtabs=false,                  % show tabs within strings adding particular underscores
  stepnumber=1,                    % the step between two line-numbers. If it's 1, each line will be numbered
  stringstyle=\color{mymauve},     % string literal style
  tabsize=4,	                   % sets default tabsize to 2 spaces
  title=\lstname                   % show the filename of files included with \lstinputlisting; also try caption instead of title
}


%opening
\title{NUR Hand-in Exercise Set 2}
\author{Aaron Seymour}

\begin{document}

\maketitle

\begin{abstract}
 My solutions to the second hand-in exercises for Numerical Recipes in Astrophysics 2019. Each question contains its function module, each subquestion (if there are subquestions) section contains its code and relevant explanations of output and plots, within and without captions.
\end{abstract}


\begin{section}{Functions Module (Exercise 1)}
% MY FUNCTIONS

Here are the functions which I call throughout the subquestions to Question 1.

\lstinputlisting{ex1functions.py}

\end{section}

\begin{section}{Exercise 1}
% EXERCISE 1(a)

\begin{subsection}{Sub-question 1(a)}

The script used to generate the results is given by:
  
\lstinputlisting{ex1a.py}

The result of the script is given by:

\lstinputlisting{ex1aoutput.txt}


\begin{figure}[h!]
  \centering
  \includegraphics[width=0.9\linewidth]{./plots/1a_uniformityanalysis.png}
  \caption{\textbf{1a}: In the \textbf{top left panel} we see the result of plotting 1000 sequential generations. There does not appear to be any correlations in the plot. The random numbers have been normalized from their nominal $2^{64}-1$ total range. By dividing by this, the output float is normalized to between 0 and 1. The \textbf{top right panel} shows the distribution of uniform draw in each iteration of the generator. You should expect to see that half of them fall about 0.5 and half below; and that there is no cycle noticeable in this generation process (i.e. we dont see the period of the generator in these 1,000 draws). The \textbf{bottom center panel} demonstrates the difference between the quality of my generator compared to an ideal uniform (flat) distribution. 1,000,000 values are generated and put into a histogram. The uncertainty of the poission process is shown as dotted lines above and below the Poissionian mean (black dotted line). The distance of each bin's height from the ideal demonstrates that only a few bins fall outside of 1-$\sigma$ of this counting process.}
  \label{fig:rngquality}
\end{figure}


\end{subsection}


\FloatBarrier
% EXERCISE 1(b)

\begin{subsection}{Sub-question 1(b)}

The script used to generate the results is given by:
  
\lstinputlisting{ex1b.py}

The Box-Muller transform takes two uniformly drawn variates and can output two standard normal variates. It works by assuming that your normally drawn variates are \textit{i.d.d} such that their product is a simple exponential of a sum that goes like:

\begin{equation}
  \propto \text{exp}(-\frac{1}{2\sigma^2}(x-\mu)^2+(y-\mu)^2)
\end{equation}

Assuming a symmetric case where $\mu$ is the mean of a 2 dimensional gaussian which is the same on both axes. This expression must be integrated in order to marginalize over the distance from the origin. Doing so is a complex integral. It can be simplified by working in polar coordinates with $x-\mu \equiv r\text{cos}\theta$ (similarly $y-\mu \equiv r\text{sin}\theta$) such that the integral reduces to the Gaussian form which has a tabulated solution.

\begin{equation}
  \propto \int_0^R\text{exp}(-\frac{-r^2}{2\sigma^2}rdr = 1-\text{exp}^{-R^2/2\sigma^2} \equiv 1 - \textit{U}
\end{equation}

The solution of this is then $R=(-2\text{ln}\textit{U})^{1/2}$ (and an angular component). These can be combined to generate standard normal variates (which can then me transformed to a new mean and standard deviation simply. See \textit{BoxMuller()} in \textit{ex1functions.py} above.)

\begin{figure}[h!]
  \centering
  \includegraphics[width=0.9\linewidth]{./plots/1b_gaussiancomparison.png}
  \caption{\textbf{1b}: A pdf normalized (multiplying by $(2\pi\sigma)^{-1/2}$) theoretical Gaussian compared to my Box-Muller transformed uniform variates. Using bins on the scale of ten times the standard deviation gives the smoothest result where both tails are populated by bins and there are not any excessively large or small bins near the mean (despite the two central bins).}
  \label{fig:gaussiancomparison}
\end{figure}



\end{subsection}

\FloatBarrier
% EXERCISE 1(c)

\begin{subsection}{Sub-question 1(c)}

The script used to generate the results is given by:
  
\lstinputlisting{ex1c.py}

The goal of a statistical test like the K-S Test is to prove that the null hypothesis; that the distribution of some data matches the proposed distribution. In this case, that means that this data (generated via the Box-Muller transformation) follows a theoretical normal distribution. The K-S test is accomplished by finding the maximum distance between Gaussian CDF and data CDF. This K-S statistic (called D) is useful because it has a known probability density function. It can be calculated as shown in Section 6.14 of Press et al., and a p-value of the signficance of the D of this observation can therefore be found from that distribution. That p-value (which will be calculated) is the probability that a value as large as D would occur if data was indeed drawn from the theoretical cdf. If the p-value is greater than or above a chosen significance level (here various levels as low as 4-$\sigma$ will be explored), then we will \texit{not} reject the hypothesis that the data come from the given distribution; if it falls below the threshold we can say it make not have been drawn from that distribution.

The KS pdf is usually calculated in terms of its cdf which is given by Press et al. as:

\begin{equation}
  1-2\sum_{j=1}^{inf}(-1)^{j-1}\text{exp}(-2j^2z^2)
\end{equation}

This converges very rapidly (as long as z>0) in about the first three terms to reach double precision accuracy.

Then the first three terms (j=1,2,3) of the complementary distribution function give a simple sum of exponential terms.

\begin{equation}
  1-2\left( \text{exp}(-2z^2) - \text{exp}(-4z^2) + \text{exp}(-18z^2)\right)
\end{equation}

\begin{figure}[h!]
  \centering
  \includegraphics[width=0.9\linewidth]{./plots/1c_kstest.png}
  \caption{\textbf{1c}: The \textbf{lower Panel} shows the result of succesive slicing of an array of $10^5$ standard normal variates is done and the resulting lists of numbers are each put through the K-S test. This takes the list, sorts it, and evaluates at each abscissa the CDF at that point. It compare how far away it lies from an actually Gaussian CDF and in each iteration over the sorted array checks to see if the current distance is larger than the previous distance. The algorithm is compared to (and matches exactly with) SciPy's KS test function. You should expect the value of D to decrease with increasing number of points because you're essentially smoothing out the rough ''corners'' of the data's CDF with increasing number of points. The \textbf{upper panel} shows the calculation of the corresponding p-value which is a function of the D-value of the lower panel. It also accounts for the number of points. The p-value here lies well above the 2-$\sigma$ level for all $10^5$ and in between. Thus my normal variates (based on my uniform number generator) agree with the null hypothesis and therefore indeed follow a standard normal distribution.}
  \label{fig:kstest}
\end{figure}


\end{subsection}

\FloatBarrier
% EXERCISE 1(d)

\begin{subsection}{Sub-question 1(d)}

The script used to generate the results is given by:
  
\lstinputlisting{ex1d.py}


The Kuiper test's statistic is V = D+ + D- is a statistic that is invariant under all shifts
and parametrizations on a circle created by ''wrapping'' around the x-axis. As such, the Kuiper will be robust against large peaks near the edges of the distribution, or in other words when the same distribution's mean is shifted towards one end of the axis.

The kuiper source code in AstroPy explains: ''Stephens 1970 claims this is more effective than the KS at detecting changes in the variance of a distribution; the KS is (he claims) more sensitive at detecting changes in the mean. D should normally be independent of the shape of CDF.''

Therefore doing the Kuiper test with a standard normal Gaussian should again accept the null hypothesis for my Box Muller generated random variates.


\begin{figure}[h!]
  \centering
  \includegraphics[width=0.9\linewidth]{./plots/1d_kuipertest.png}
  \caption{\textbf{1d}: Again, the \textbf{lower Panel} shows the result of succesive slicing of an array of $10^5$ standard normal variates, except this time the Kuiper Test is performed. Similarly, it sorts it, and evaluates the CDF but in each iteration stores both the largest negative and largest positive distances instead. The algorithm is again also compared to SciPy's KS test function (modified in calling to return separate D+ and D- statistics, see the below figure for these two statistics separately) D should again decrease with increasing number of points. The \textbf{upper panel} shows the calculation of the corresponding p-values. In my algorithm, the p-value seems to dip below the 2-$\sigma$ level for intermediate numbers of points, whereas SciPy's value's do not. Although this only happens for exactly 1,000 points and no where else. Scipy's test still says that the null hypothesis is accepted and my data follows a standard normal distribution.}
  \label{fig:kuipertest}
\end{figure}



\begin{figure}[h!]
  \centering
  \includegraphics[width=0.9\linewidth]{./plots/1d_kuipertest_lohi.png}
  \caption{\textbf{1d}: A comparison of Scipy's KS and my Kuiper test algorithms for the two statistics, D+ and D-. They both do decrease with increasing number of points and are similar.}
  \label{fig:kuipertest2}
\end{figure}


\end{subsection}


\FloatBarrier
% EXERCISE 1(e)

\begin{subsection}{Sub-question 1(e)}

The script used to generate the results is given by:
  
\lstinputlisting{ex1e.py}

A p-value will be used to determine if these data sets can be accepted in light of the null hypothesis that they follow a standard normal distribution. For those data sets that do not abide by the null hypothesis, we should expect that the p-value will fall below at least the 2-$\sigma$ level when the number of points gets high enough. 


\begin{figure}[h!]
  \centering
  \includegraphics[width=0.8\linewidth]{./plots/1e_kuipertest_data1.png}
  \caption{\textbf{1e}: Despite the value of the V statistic decreasing with increasing number of points as is expected, the p-value for accepting the null hypothesis that this data set follows ( or is drawn from) a standard normal distribution falls off quickly with number of points.}
\end{figure}

\begin{figure}[h!]
  \centering
  \includegraphics[width=0.8\linewidth]{./plots/1e_kuipertest_data2.png}
  \caption{\textbf{1e}: An apparently uniform distribution is definitely not standard normal.}
\end{figure}

\begin{figure}[h!]
  \centering
  \includegraphics[width=0.8\linewidth]{./plots/1e_kuipertest_data3.png}
  \caption{\textbf{1e}: }
\end{figure}

\begin{figure}[h!]
  \centering
  \includegraphics[width=0.8\linewidth]{./plots/1e_kuipertest_data4.png}
  \caption{\textbf{1e}: Definitely passes the Kuiper test and so follows a standard normal distribution.}
\end{figure}

\begin{figure}[h!]
  \centering
  \includegraphics[width=0.8\linewidth]{./plots/1e_kuipertest_data5.png}
  \caption{\textbf{1e}: This distribution is too flat everywhere to be drawn from a standard normal distribution.}
\end{figure}

\begin{figure}[h!]
  \centering
  \includegraphics[width=0.8\linewidth]{./plots/1e_kuipertest_data6.png}
  \caption{\textbf{1e}: Resembles the null hypothesis despite strange discontinuities at a couple points along the x-axis.}
\end{figure}

\begin{figure}[h!]
  \centering
  \includegraphics[width=0.8\linewidth]{./plots/1e_kuipertest_data7.png}
  \caption{\textbf{1e}: Visually appears to be the most Gaussian, but is too flat at the peak and so fails after 1000 points.}
\end{figure}

\begin{figure}[h!]
  \centering
  \includegraphics[width=0.8\linewidth]{./plots/1e_kuipertest_data8.png}
  \caption{\textbf{1e}: }
\end{figure}

\begin{figure}[h!]
  \centering
  \includegraphics[width=0.8\linewidth]{./plots/1e_kuipertest_data9.png}
  \caption{\textbf{1e}: }
\end{figure}

\begin{figure}[h!]
  \centering
  \includegraphics[width=0.8\linewidth]{./plots/1e_kuipertest_data10.png}
  \caption{\textbf{1e}: This data set has a large negative skew which causes it to fail.}
\end{figure}

It is apparent from these plots that it takes about 100 points in order for the p-value to fall below the 2-$\sigma$ level. And none survive past 500 points if they are truly not standard normal. Those that do pass the Kuiper test are data sets 4 and 6. They also resemble the plots from the previous sub question, where the p-value remained above the 2-$\sigma$ level, despite some fluctation. I am unsure why data set six diveges to 0 and then returns at the next point to above 2-$\sigma$; this could be due to a discontinuity in the slicing of my random normal rvs, since it also does not survive up past about $10^{4.5}$ points.

\end{subsection}

\end{section}


\FloatBarrier
\begin{section}{Functions Module (Exercise 2)}
% MY FUNCTIONS

Here are the functions which I call throughout the subquestions to Question 2.

\lstinputlisting{ex2functions.py}

\end{section}

\begin{section}{Exercise 2}
The script used to generate the results is given by:

\lstinputlisting{ex2.py}


For convenience in the DFT, an even grid size is take of N = 1024 pixels on one side. This always leaves the Nyquist frequency in the complex Fourier plane of fourier coefficients in the the center of the image. The grid side length defines the sampling interval $1/N$, so that the Nyquist frequency in this case becomes $(1/2)(1/N) = N/2$.

Therefore, a grid is initialized with those fourier $k$ coefficients starting from 0:

\begin{equation}
    k_x=(0,1,2, ... , N/2, 1- (N/2), ... , -3,-2,-1)
\end{equation}

Doing the same for the $k_y$ component, and creating a matrix of with ($k_x$,$k_y$) indices will satisfy the condition of a symmetric complex fourier plane.

\begin{figure}[h!]
  \centering
  \includegraphics[width=0.9\linewidth]{./plots/2_kspace.png}
  \caption{\textbf{2}: The result space of k values for the above sequence. The symmetric matrix of indices is constructed by drawing from the above 1D sequence at corresponding points in the grid defined by N. These symmetric $k_x$ and $k_y$ (a function of n in $k^n$) values are used to draw variates.}
  \label{fig:kspace}
\end{figure}

The above series stops at -1 because these is no ''negative zero frequency.'' At each point in this matrix, two Gaussian variates are draw with standard deviation $(k_x^2+k_y^2)^{n/2}$ corresponding to the wave number dependent dispersion which goes as $k^n$.

The resulting matrix of complex vectors is then Rayleigh distributed. Taking its inverse DFT gives a Gaussian random field. The power spectrum of this Gaussain field can be modified by changing the value of $n$ above.

\begin{figure}[h!]
  \centering
  \includegraphics[width=0.9\linewidth]{./plots/2_fourier_and_realplane.png}
  \caption{\textbf{1d}: Taking the absolute value of the two complex matrices. The \textbf{left column} is the absolute value of the fourier plane constructed of drawn normal variates with variance that goes with the 2D wave number indices; this matrix is by definition complex since each point in the grid is constructed as \textit{a+bi} with a,b drawn variates. The \textbf{right column} is the absolute value of the inverse fourier transform of that matrix. It is still necessary to take the absolute value since the (remaining) imaginary component.}
  \label{fig:abs}
\end{figure}


\begin{figure}[h!]
  \centering
  \includegraphics[width=0.9\linewidth]{./plots/2_realplane_imagrealparts.png}
  \caption{\textbf{2}: The imaginary part in the \textbf{right column} should be close to zero having satisfied the symmetry in the array of fourier amplitude coefficients: when $\widetilde{Y}(-k)=\widetilde{Y}*(k)$ is satisfied, then the output inverse DFT should ideally be entirely real valued (no imaginary component left.) Due to machine floating point precision, there is a remaining imaginary component. The imaginary components should therefore be very close to zero, or very ''dim'' when scaled to the same color bar as the \textbf{left column} real components. That seems to not always hold true upon the generation of different fields (with different seeding.)}
  \label{fig:imagrealparts}
\end{figure}



\end{section}


\FloatBarrier
\begin{section}{Functions Module (Exercise 3)}
% MY FUNCTIONS

Here are the functions which I call throughout the subquestions to Question 3.

\lstinputlisting{ex3functions.py}


\end{section}

\begin{section}{Exercise 3}
The script used to generate the results is given by:

\lstinputlisting{ex3.py}

In an Einstein-de Sitter Universe where the only component in the equations of state is matter with density $\Omega_m$=1 (and so is also spatially flat), the scale factor can be found via the Friedmann Equation as 2/3 power law in time.

By separating the temporal and spatial components in the Second O.D.E. of density perturbations $\delta = D(t)\Delta(x)$, the differential equation can be written in just the spatial component as:

\begin{equation}
  \frac{d^2D}{dt^2}+2\frac{\dot{a}}{a}\frac{dD}{dt}=\frac{3}{2}\Omega_0H_0^2\frac{1}{a^3}D
\end{equation}

This is the linearized density growth equation. Plugging in the scale factor (where the total density $\Omega_0 \rightarrow \Omega_m$):

\begin{equation}
  \frac{\delta^2D}{\delta t^2}=-\frac{4}{3t}\frac{\delta D}{\delta t}+\frac{2\Omega_m}{3t^2}D
\end{equation}

Meaning this second order differential equation can be written in the form

\begin{equation}
  \frac{\delta^2D}{\delta t^2}=f\left(t,D,\frac{\delta D}{\delta t}\right)
\end{equation}

Something like Euler's method for integrating ODE's leads to a lot of error (on the order of the step size). This is due to local truncations adding up in each step of the numerical integration. In this case Euler's method would probably suffice given the smoothness of the analytical solution(s). In general, a higher order method such as Runge Kutta mitigates this truncation error by stepping to the next point based on a weighted average of slopes at the midpoint between steps. Fourth order RK gives the most computationally cost efficient solution, and succeeds almost always. That is why RK4 was chosen over simple Euler's.

To use RK4, a second order differential equation like this one must be rewritten as as two coupled first order differential equations. Therefore, define:

\begin{equation}
  \frac{\delta D}{\delta t} \equiv z
\end{equation}

This is just a first order differential equation which RK4 is very good at solving as described above. With the above definition and the simplified Einstein-de Sitter spatial density perturbation equation, this leaves the coupled set of first order differential equations:

\begin{align*}
  \frac{\delta D}{\delta t} &= g(t,D,z)\\
  \frac{\delta z}{\delta t} &= f(t,D,z)
\end{align*}


This system can be solved for using the RK4 method in order to integrate the original second order differential equation and find the spatial density growth term.

Analytically, we can find the general solution to the second order differential equation above as:

\begin{equation}
  D(t) = c_1t^{2/3}+c_2/t
\end{equation}

aSolving for the coefficients using the initial values this leaves the solutions:

\begin{itemize}
  \item Case 1: $D(t) = 3t^{2/3}$ 
  \item Case 2: $D(t) =10/t$
  \item Case 3: $D(t) =  3t^{2/3}+ 2/t$
  \end{itemize}



\begin{figure}[h!]
  \centering
  \includegraphics[width=0.9\linewidth]{./plots/3_spatialdensitygrowth.png}
  \caption{\textbf{3}: The three provided cases of spatial density growth. Case 1 and 3 are growing modes which means the spatial density is increasing. Case 2 is a decaying mode. The analytical solutions shown above have been overplotted on the RK4 solutions down in my code. We can see good agreement between the two; there are only differences on the order of $10^{-9}$.}
  \label{fig:spatialdensitygrowth}
\end{figure}


\begin{figure}[h!]
  \centering
  \includegraphics[width=0.9\linewidth]{./plots/3_spatialdensitygrowth_residuals.png}
  \caption{\textbf{3}: The residuals between the analytical solutions with solved coefficients and my RK4 solutions show agreement to $10^{-9}$.}
  \label{fig:spatialdensitygrowthresiduals}
\end{figure}

The first time derivative term ($\propto (\dot{a}/a)(\delta D /\delta t)$) can be interpretted as a frictional term (sometimes called ''Hubble friction''). This means that its sign determines the behavior of the spatial growth of structure, in terms of expansion or collapse.


\begin{figure}[h!]
  \centering
  \includegraphics[width=0.9\linewidth]{./plots/3_spatialdensitygrowth_near_t0.png}
  \caption{\textbf{3}: Again showing the spatial growth at early times.}
  \label{fig:spatialdensitygrowtht0}
\end{figure}


The plots show two important things: growing modes (Case 1 \& Case 3) and a decaying mode (Case 2). These modes are important in analyzing the formation of structure of the early universe. In this matter dominated universe, over dense regions expand less rapidly than elsewhere. Gravitational instability causes initial growth perturbations to collapse to high density and form galaxy clusters.

The linearized perturbations growth temporal term $D(t)$ characterizes how these distributions form with time. Case 2 has a continuously declining density: the contrast in density between regions diminishes with time. At late times, it can be discarded from further considerations in the evolution of structure. A negative $D'(1)$ term prevents increasing density as a pressure against gravitational collapse. The opposite condition is represented in Case 1 where a small initial perturbation given by $D(1)$ (characterized also in the Jeans Length) and a small positive pressure allow pertubations to grow and form clusters for all time. Case 3 is somewhere between these two where it initially looks like it could expand or collapse at very early times, but the perturbations lead to exponential collapse after a few tens of years; its initial density is too low to overcome gravity.

%https://www.astro.rug.nl/~weygaert/tim1publication/lss2009/lss2009.linperturb.pdf
%https://math.stackexchange.com/questions/721076/help-with-using-the-runge-kutta-4th-order-method-on-a-system-of-2-first-order-od
\end{section}


\FloatBarrier
\begin{section}{Functions Module (Exercise 4)}
% MY FUNCTIONS

Here are the functions which I call throughout the subquestions to Question 4.

\lstinputlisting{ex4functions.py}

\end{section}

\begin{section}{Exercise 4}
\begin{subsection}{Sub-question 4(a)}

The script used to generate the results is given by:

\lstinputlisting{ex4.py}


The result of the script is given by:

\lstinputlisting{ex4output.txt}

\end{subsection}


\begin{subsection}{Sub-question 4(b)}
  
To find the time derivative of the linear growth factor:

\begin{align}
  \dot{D}(t) &= \frac{dD}{da}\dot{a} \\
             &= \frac{dD}{da}H(z)a(z)\\
             &= \frac{dD}{da}\left[H_0^2(\Omega_m(1+z)^3+\Omega_\Lambda)\right]^{1/2}\frac{1}{1+z}\\
\end{align}

At a single value of z, we can approximate $dD/da$ as roughly $D/a$, also known as the linear growth factor. Then using the same definition of redshift as above ($a=1/1+z$):

\begin{align}
  \dot{D}(t) &\approx (D)(1+z)\left[H_0^2(\Omega_m(1+z)^3+\Omega_\Lambda)\right]^{1/2}\frac{1}{1+z}\\
             &=(D)\left[H_0^2(\Omega_m(1+z)^3+\Omega_\Lambda)\right]^{1/2}\\
\end{align}

Then we can plug in all the known values (where $D$ was found at z=50 in part (a):



\begin{equation}
  \dot{D}(50) = (0.0196078)(7.16e-11/yr)(0.3(1+50)^3+0.7) = 5.6 \times 10^{-8}
\end{equation}

\end{subsection}
\end{section}


\FloatBarrier
\begin{section}{Functions Module (Exercise 6)}
% MY FUNCTIONS

Here are the functions which I call throughout the subquestions to Question 6.

\lstinputlisting{ex6functions.py}


\end{section}

\begin{section}{Exercise 6}

The script used to generate the results is given by:

\lstinputlisting{ex6.py}

The result of the script is given by:

\lstinputlisting{ex6output.txt}



A classification problem involves labeling a set of features in a data set and choosing which features have the most effect on the behavior of the desired classification item. To do this we can plot each feature against each other. A feature is then excluded based on when one feature exhibits no variation as a function of the other the whole range their values. Without variation, a feature will not add anything to the success of the training.

The features with the most useful behavior after plotting them all against eachother are shown below. The best behavior is reshift against log mass. It has the greatest quantity of samples remaining after dealing with the missing data.
\begin{figure}[h!]
  \centering
  \includegraphics[width=0.9\linewidth]{./plots/6_featurecomparison.png}
  \caption{\textbf{6}: The most desireable set of features has been highlighted in gold. It exhibits a significantly large cloud of points away from the missing data which itself has aspatial dependence on the GRB type. (The $T_{90}$ feature has been exlcuded from the above choice of feature removal as this  is the quantity we will be using to label the data set and so it cannot be used to train on.)}
  \label{fig:featurecomparison}
\end{figure}

The SSFR feature shows nearly no behavior when plotted against all the other features. Whereas the redshift versus log($M*/M_{\odot}$) plot has an interesting cloud of points which itself depends on if GRB is short or long.

To perform the logistic regression to classify these GRB's on redshift, log mass, and star formation rate, we first need to label the data set by the value of $T_{90}$ based on the 10 second threshold known to identify short from long GRBs. Next, we can solve the problem of missing data by setting all the missing data points to zero. In this way, any associated weights given to those samples of an included feature which have missing data just drop out of the regression.

Just prior to the actual training via gradient descent to find parameters the features must be scaled. This speeds up the training by making the gradient descent phase converge must faster since mostly all the parameters will be made to lie in the range [-1,1].

(At this point we can choose to place these two features in a basic 1D polynomial, just a linear combination, or a higher order 2D polynomial. The distinction comes in proper fitting. If the order of this polynomial goes much higher than 2, then we risk over fitting. But we also risk under fitting by only taking the simple linear combination of two features.)

Parameters and a bias are intialized to zero (giving three parameters in the case of a bias, and two features in linear combination.)

Logistic regression is based on the formation of a hypothesis, the calculation of a loss function which is a function of how far away that prediction is from the known label, and ultimately a cost function which is what is minimized. This cost function is a function of the loss and is just a sum of the loss functions over all the samples of a feature. The cost in this case in therefore a single value which can be minimized if the parameters (or weights) of each feature are found such that their combination most closely matches the known label (so that the hypothesis for a sample is close to its label and the loss function will be small.)

Gradient descent (GD) is performed while the error (a difference between successive cost functions) remains above a desired thresholh value. Here that is set a $10^{-6}$, but initialized as a very large value. GD takes those parameters, and performs the above described sequence of calculations and determines slightly alters the parameter values in the hopes of minimizing the cost in each iteration. With successful GD, the error should converge down to the threshold and the cost function should also converge down to some value.

\begin{figure}[h!]
  \centering
  \includegraphics[width=0.9\linewidth]{./plots/6_classificationperformance.png}
  \caption{\textbf{6}: The cost function which sums up how how the hypothesis (based on three parameters and a bias) differs from the labels in each iteration converges down to a minimum after a couple thousand iterations of the gradient descent given the setting of the learning rate to <0.1.}
  \label{fig:classificationperformance}
\end{figure}

The success of the training (to find the proper weights for each samples' features) can be determined via the activation function, which is the sigmoid in this case. Applying the sigmoid function to the dot product of the weights and features (including a the bias) gives the result of what the training algorithm determine to be a short or long GRB (by outputting a value between 0 and 1, whereas the labels are 0 or 1.)

\begin{equation}
  h_{\theta}(x^{(i)}) \equiv \hat{y} = \sigma(\theta^Tx) = \sigma(\theta_0+\theta_1x_i^{(i)}+\theta_2x_i^{(i)}+\theta_3x_i^{(i)})
\end{equation}

Further, the F1-score quantifies the accuracy of the identifications the classifier makes given the known labels for each GRB sample.

In this case, the classifier reaches an F1 score of almost 60\%. I tried taking the hypothesis without and without the bias weight. Using the bias seems to make the classifier identify all of the long GRBs but none of the short GRBs. This is strange behavior which gives an F1 score of almost 90\%. This does not seem correct. If I instead exclude the bias, in the output predictions, there are about half of each type of GRB identified (out of 185 long and 50 short GRBs.)

A decision boundary can be constructed based on the line formed by the linear combination of features. It is the boundary where the hypothesis is equal to 0.5, such either side of the line represents where the hypothesis was larger or smaller than 0.5. This linear combination is just the equation for a line which has an ''x-intercept'', a ''y-intercept'', and a slope which can be solved for and plotted.

\begin{equation}
  x_2 = -\frac{\theta_1}{\theta_2}x_1 + \frac{\text{logit}(0.5)-\theta_0}{\theta_2}
\end{equation}

Where $x_1$ and $x_2$ are the training samples, $\theta_0$ is the bias input, $\theta_1$, and $\theta_2$ are the feature weights. The logit term is 0 here for a decision boundary at $\hat{y}$=0.5, the boundary between the classification of short and long GRBs can be found for different thresholds of identificaiton.

\begin{figure}[h!]
  \centering
  \includegraphics[width=0.9\linewidth]{./plots/6_decisionboundary_x1x2.png}
  \caption{\textbf{6}: The plots shows that there are two types of GRBs, blue dots for short, red dots for long. The 0.5 decision boundary should be a line dividing the two classes as best as possible based on the training. These boundaries are showing unexpected behavior which may be due to a plotting error or an unaccounted sign somewhere in the code. I don't think its a misinterpretation of my trained weights.}
  \label{fig:decisionboundary1}
\end{figure}

\begin{figure}[h!]
  \centering
  \includegraphics[width=0.9\linewidth]{./plots/6_decisionboundary_x2x3.png}
  %\caption{\textbf{6}: The plots shows that there are two types of GRBs, blue dots for short, red dots for long. The 0.5 decision boundary should be a line dividing the two classes as best as possible based on the training. These boundaries are showing unexpected behavior which may be due to a plotting error or an unaccounted sign somewhere in the code. I don't think its a misinterpretation of my trained weights.}
  \label{fig:decisionboundary2}
\end{figure}
\end{section}


\FloatBarrier
\begin{section}{Exercise 7}
The script used to generate the results is given by:
  
\lstinputlisting{ex7.py}

The result of the script is given by:

\lstinputlisting{ex7output.txt}


With a root node composed of children, which are each composed of children down to the leaf node (containing ideally just one coordinate point), the Barnes Hut Algorithm can be appliedto find the center of mass of sections of space. Here, only a minimum of 12 coordinates per node is required. This can be achieved in 2 manual (as opposed to recursive) iterations of the quadtree building.


\begin{figure}[h!]
  \centering
  \includegraphics[width=0.9\linewidth]{./plots/7_150particles1stiter.png}
  \caption{\textbf{7}: The Barnes Hut Quadtree with initial nodes denoted by color in each corner. The particle of interest has been starred.}
  \label{fig:spatialdensitygrowtht0}
\end{figure}


In the B.H. algorithm, groups of points are succesively merged via calculating their multipole moment. The n=0 multipole moment is just the point mass approximate:

\begin{equation}
  M_0=\sum_i m_a
\end{equation}

Since all these particle masses are equal to 0.0125, the zeroeth multiple moments will only differ by a multiple of that mass. This also means that you get back just the total mass of all the particles upon adding up all the nodes. But the node moments won't be equal as they have different amounts of particles in them (150 in this case).


\begin{figure}[h!]
  \centering
  \includegraphics[width=0.9\linewidth]{./plots/7_150particles2nditer.png}
  \caption{\textbf{7}: The Barnes Hut Quadtree with leaf nodes colored. These leaf nodes contain 12 particles or less and are subsets of the nodes above. The zeroeth multipole moment is found in each of the colored leafnodes (by summing up the mass.) Then the multipole moment of the parent node, and eventually root node can be found in the same way.}
\end{figure}
\end{section}
  



\end{document}
