\documentclass[a4paper,10pt]{article}
\usepackage[utf8]{inputenc}
\usepackage{placeins}
\input{included_packages}

%opening
\title{NUR Hand-in Exercise Set 2}
\author{Aaron Seymour}

\begin{document}

\maketitle

\begin{abstract}
 My solutions to the second hand-in exercises for Numerical Recipes in Astrophysics 2019. Each question contains its function module, each subquestion (if there are subquestions) section contains its code and relevant explanations of output and plots, within and without captions.
\end{abstract}


\begin{section}{Functions Module (Exercise 1)}
% MY FUNCTIONS

Here are the functions which I call throughout the subquestions to Question 1.

\lstinputlisting{ex1functions.py}

\end{section}

\begin{section}{Exercise 1}
% EXERCISE 1(a)

\begin{subsection}{1(a)}
  
\lstinputlisting{ex1a.py}

The KS pdf is usually calculated in terms of its cdf which is given by Press et al.

\begin{equation}
  1-2\sum_{j=1}^{inf}(-1)^{j-1}\text{exp}(-2j^2z^2)
\end{equation}

This converges very rapidly (as long as z>0) in about the first three terms to reach double precision accuracy.

Then the first three terms (j=1,2,3) of the complementary distribution function give a simple sum of exponential terms.

\begin{equation}
  1-2\left( \text{exp}(-2z^2) - \text{exp}(-4z^2) + \text{exp}(-18z^2)\right)
\end{equation}

\end{subsection}


% EXERCISE 1(b)

\begin{subsection}{1(b)}
  
\lstinputlisting{ex1b.py}


\end{subsection}


% EXERCISE 1(c)

\begin{subsection}{1(c)}
  
\lstinputlisting{ex1c.py}


\end{subsection}


% EXERCISE 1(d)

\begin{subsection}{1(d)}
  
\lstinputlisting{ex1d.py}


\end{subsection}

% EXERCISE 1(e)

\begin{subsection}{1(e)}
  
\lstinputlisting{ex1e.py}


\end{subsection}

\end{section}


\FloatBarrier
\begin{section}{Functions Module (Exercise 2)}
% MY FUNCTIONS

Here are the functions which I call throughout the subquestions to Question 2.

\lstinputlisting{ex2functions.py}

\end{section}

\begin{section}{Exercise 2}
The script used to generate the results is given by:

\lstinputlisting{ex2.py}


For convenience in the DFT, an even grid size is take of N = 1024 pixels on one side. This always leaves the Nyquist frequency in the complex Fourier plane of fourier coefficients in the the center of the image. The grid side length defines the sampling interval $1/N$, so that the Nyquist frequency in this case becomes $(1/2)(1/N) = N/2$.

Therefore, a grid is initialized with those fourier $k$ coefficients starting from 0:

\begin{equation}
    k_x=(0,1,2, ... , N/2, 1- (N/2), ... , -3,-2,-1)
\end{equation}

Doing the same for the $k_y$ component, and creating a matrix of with ($k_x$,$k_y$) indices will satisfy the condition of obtaining a real inverse fourier transform: conjugate symmetry of fourier coefficients.

\begin{figure}[h!]
  \centering
  \includegraphics[width=0.9\linewidth]{./plots/2_kspace.png}
  \caption{\textbf{2}: The result space of k values for the above sequence. The symmetric matrix of indices is constructed by drawing from the above 1D sequence at corresponding points in the grid defined by N. These symmetric $k_x$ and $k_y$ (a function of n in $k^n$) values are used to draw variates.}
  \label{fig:kspace}
\end{figure}

The above series stops at -1 because these is no ''negative zero frequency.'' At each point in this matrix, two \textit{i.i.d.} Gaussian variates are draw with a standard deviation $(k_x^2+k_y^2)^{n/2}$ corresponding to the wave number dependent dispersion which goes as $k^n$.

The resulting matrix of complex vectors is then Rayleigh distributed. Taking its inverse DFT gives a Gaussian random field. The power spectrum of this Gaussain field can be modified by changing the value of $n$ above.

\begin{figure}[h!]
  \centering
  \includegraphics[width=0.9\linewidth]{./plots/2_fourier_and_realplane.png}
  \caption{\textbf{1d}: Taking the absolute value of the two complex matrices. The \textbf{left column} is the absolute value of the fourier plane constructed of drawn normal variates with variance that goes with the 2D wave number indices; this matrix is by definition complex since each point in the grid is constructed as \textit{a+bi} with a,b drawn variates. The \textbf{right column} is the absolute value of the inverse fourier transform of that matrix. It is still necessary to take the absolute value since the (remaining) imaginary component.}
  \label{fig:abs}
\end{figure}


\begin{figure}[h!]
  \centering
  \includegraphics[width=0.9\linewidth]{./plots/2_realplane_imagrealparts.png}
  \caption{\textbf{2}: The imaginary part in the \textbf{right column} should be close to zero having satisfied the symmetry in the array of fourier amplitude coefficients: when $\widetilde{Y}(-k)=\widetilde{Y}*(k)$ is satisfied, then the output inverse DFT should ideally be entirely real valued (no imaginary component left.) Due to machine floating point precision, there is a remaining imaginary component. The imaginary components should therefore be very close to zero, or very ''dim'' when scaled to the same color bar as the \textbf{left column} real components. That seems to not always hold true upon the generation of different fields (with different seeding.)}
  \label{fig:imagrealparts}
\end{figure}



\end{section}


\FloatBarrier
\begin{section}{Functions Module (Exercise 3)}
% MY FUNCTIONS

Here are the functions which I call throughout the subquestions to Question 3.

\lstinputlisting{ex3functions.py}


\end{section}

\begin{section}{Exercise 3}
The script used to generate the results is given by:

\lstinputlisting{ex3.py}

In an Einstein-de Sitter Universe where the only component in the equations of state is matter with density $\Omega_m$=1 (and so is also spatially flat), the scale factor can be found via the Friedmann Equation as 2/3 power law in time.

By separating the temporal and spatial components in the Second O.D.E. of density perturbations $\delta = D(t)\Delta(x)$, the differential equation can be written in just the spatial component as:

\begin{equation}
  \frac{d^2D}{dt^2}+2\frac{\dot{a}}{a}\frac{dD}{dt}=\frac{3}{2}\Omega_0H_0^2\frac{1}{a^3}D
\end{equation}

This is the linearized density growth equation. Plugging in the scale factor (where the total density $\Omega_0 \rightarrow \Omega_m$):

\begin{equation}
  \frac{\delta^2D}{\delta t^2}=-\frac{4}{3t}\frac{\delta D}{\delta t}+\frac{2\Omega_m}{3t^2}D
\end{equation}

Meaning this second order differential equation can be written in the form

\begin{equation}
  \frac{\delta^2D}{\delta t^2}=f\left(t,D,\frac{\delta D}{\delta t}\right)
\end{equation}

Something like Euler's method for integrating ODE's leads to a lot of error (on the order of the step size). This is due to local truncations adding up in each step of the numerical integration. In this case Euler's method would probably suffice given the smoothness of the analytical solution(s). In general, a higher order method such as Runge Kutta mitigates this truncation error by stepping to the next point based on a weighted average of slopes at the midpoint between steps. Fourth order RK gives the most computationally cost efficient solution, and succeeds almost always. That is why RK4 was chosen over simple Euler's.

To use RK4, a second order differential equation like this one must be rewritten as as two coupled first order differential equations. Therefore, define:

\begin{equation}
  \frac{\delta D}{\delta t} \equiv z
\end{equation}

This is just a first order differential equation which RK4 is very good at solving as described above. With the above definition and the simplified Einstein-de Sitter spatial density perturbation equation, this leaves the coupled set of first order differential equations:

\begin{align*}
  \frac{\delta D}{\delta t} &= g(t,D,z)\\
  \frac{\delta z}{\delta t} &= f(t,D,z)
\end{align*}


This system can be solved for using the RK4 method in order to integrate the original second order differential equation and find the spatial density growth term.

Analytically, we can find the general solution to the second order differential equation above as:

\begin{equation}
  D(t) = c_1t^{2/3}+c_2/t
\end{equation}

aSolving for the coefficients using the initial values this leaves the solutions:

\begin{itemize}
  \item Case 1: $D(t) = 3t^{2/3}$ 
  \item Case 2: $D(t) =10/t$
  \item Case 3: $D(t) =  3t^{2/3}+ 2/t$
  \end{itemize}



\begin{figure}[h!]
  \centering
  \includegraphics[width=0.9\linewidth]{./plots/3_spatialdensitygrowth.png}
  \caption{\textbf{3}: The three provided cases of spatial density growth. Case 1 and 3 are growing modes which means the spatial density is increasing. Case 2 is a decaying mode. The analytical solutions shown above have been overplotted on the RK4 solutions down in my code. We can see good agreement between the two; there are only differences on the order of $10^{-9}$.}
  \label{fig:spatialdensitygrowth}
\end{figure}


\begin{figure}[h!]
  \centering
  \includegraphics[width=0.9\linewidth]{./plots/3_spatialdensitygrowth_residuals.png}
  \caption{\textbf{3}: The residuals between the analytical solutions with solved coefficients and my RK4 solutions show agreement to $10^{-9}$.}
  \label{fig:spatialdensitygrowthresiduals}
\end{figure}

The first time derivative term ($\propto (\dot{a}/a)(\delta D /\delta t)$) can be interpretted as a frictional term (sometimes called ''Hubble friction''). This means that its sign determines the behavior of the spatial growth of structure, in terms of expansion or collapse.


\begin{figure}[h!]
  \centering
  \includegraphics[width=0.9\linewidth]{./plots/3_spatialdensitygrowth_near_t0.png}
  \caption{\textbf{3}: Again showing the spatial growth at early times.}
  \label{fig:spatialdensitygrowtht0}
\end{figure}


The plots show two important things: growing modes (Case 1 \& Case 3) and a decaying mode (Case 2). These modes are important in analyzing the formation of structure of the early universe. In this matter dominated universe, over dense regions expand less rapidly than elsewhere. Gravitational instability causes initial growth perturbations to collapse to high density and form galaxy clusters.

The linearized perturbations growth temporal term $D(t)$ characterizes how these distributions form with time. Case 2 has a continuously declining density: the contrast in density between regions diminishes with time. At late times, it can be discarded from further considerations in the evolution of structure. A negative $D'(1)$ term prevents increasing density as a pressure against gravitational collapse. The opposite condition is represented in Case 1 where a small initial perturbation given by $D(1)$ (characterized also in the Jeans Length) and a small positive pressure allow pertubations to grow and form clusters for all time. Case 3 is somewhere between these two where it initially looks like it could expand or collapse at very early times, but the perturbations lead to exponential collapse after a few tens of years; its initial density is too low to overcome gravity.

%https://www.astro.rug.nl/~weygaert/tim1publication/lss2009/lss2009.linperturb.pdf
%https://math.stackexchange.com/questions/721076/help-with-using-the-runge-kutta-4th-order-method-on-a-system-of-2-first-order-od
\end{section}


\FloatBarrier
\begin{section}{Functions Module (Exercise 4)}
% MY FUNCTIONS

Here are the functions which I call throughout the subquestions to Question 4.

\lstinputlisting{ex4functions.py}

\end{section}

\begin{section}{Exercise 4}
The script used to generate the results is given by:

\lstinputlisting{ex4.py}


The result of the script is given by:

\lstinputlisting{ex6output.txt}


To find the time derivative of the linear growth factor:

\begin{align}
  \dot{D}(t) &= \frac{dD}{da}\dot{a} \\
             &= \frac{dD}{da}H(z)a(z)\\
             &= \frac{dD}{da}\left[H_0^2(\Omega_m(1+z)^3+\Omega_\Lambda)\right]^{1/2}\frac{1}{1+z}\\
\end{align}

At a single value of z, we can approximate $dD/da$ as roughly $D/a$, also known as the linear growth factor. Then using the same definition of redshift as above ($a=1/1+z$):

\begin{align}
  \dot{D}(t) &\approx (D)(1+z)\left[H_0^2(\Omega_m(1+z)^3+\Omega_\Lambda)\right]^{1/2}\frac{1}{1+z}\\
             &=(D)\left[H_0^2(\Omega_m(1+z)^3+\Omega_\Lambda)\right]^{1/2}\\
\end{align}

Then we can plug in all the known values (where $D$ was found at z=50 in part (a):

Om_m=0.3,Om_L=0.7,H_0=7.16e-11

\begin{equation}
  \dot{D}(50) = (0.0196078)(7.16e-11/yr)(0.3(1+50)^3+0.7) = 5.6 \times 10^{-8}
\end{equation}


\end{section}


\begin{section}{Exercise 5}
The script used to generate the results is given by:
  
\lstinputlisting{ex5.py}


The above script is my attempt at the solution. It fails to get past the recursive call to create the even frequency components. But it successfully does bit reversals of each slicing of the input array. If it worked, it would create the even and odd component arrays and then place those into a DFT (''split1'' and ''split2'') in order to reconstruct the final N frequency components of the FFT, from the original N time/spatial components of the input data.

An FFT does the same summation that a DFT does, but instead of $N^2$ operations, the FFT does $N\text{log}_2N$ operations.

In an algorithm like Cooley-Tukey, the DFT sum is successively split into two halves. Making sure N is even integer (or else it is padded) leaves two DFTs in N/2 of the original points. This is further split down to single element sums. This algorithm factorizes the DFT data array into a product of mostly zero factors in the end, making it very fast.
\end{section}


\FloatBarrier
\begin{section}{Functions Module (Exercise 6)}
% MY FUNCTIONS

Here are the functions which I call throughout the subquestions to Question 6.

\lstinputlisting{ex6functions.py}


\end{section}

\begin{section}{Exercise 6}

The script used to generate the results is given by:

\lstinputlisting{ex6.py}

The result of the script is given by:

\lstinputlisting{ex6output.txt}



A classification problem involves labeling a set of features in a data set and choosing which features have the most effect on the behavior of the desired classification item. To do this we can plot each feature against each other. A feature is then excluded based on when one feature exhibits no variation as a function of the other the whole range their values. Without variation, a feature will probably not add anything to the success of the training.


\begin{figure}[h!]
  \centering
  \includegraphics[width=0.9\linewidth]{./plots/6_featurecomparison.png}
  \caption{\textbf{6}: The most desireable set of features has been highlighted in gold. It exhibits a significantly large cloud of points away from the missing data which is also split by GRB type. (The $T_{90}$ feature has been exlcuded from the above plot as this  is the quantity we will be using to label the data set and so it cannot be used to train on.)}
  \label{fig:featurecomparison}
\end{figure}

The features that which show the least amount of interesting clustering behavior in the plots are the SFR, log($Z/Z_{\odot}$), SSFR, and the AV. Whereas the redshift shows and log($M*/M_{\odot}$) have an interesting cloud of points away from zero which and within that depends on if GRB is short or long.

To perform the logistic regression to classify these GRB's on redshift and log mass, we first need to label the data set by the value of $T_{90}$ based on the 10 second threshold known to identify short from long GRBs. Next, we can solve the problem of missing data by setting all the missing data points to zero. In this way, any associated weights given to those samples of an included feature which have missing data just drop out of the regression. The all but the above mentioned two features are excluded from this training.

Just prior to the actual training, involving gradient descent to find parameters, these features must be scaled. This speeds up the training by making the gradient descent phase converge must faster since mostly all the parameters will be made to lie in the range [-1,1].

(At this point we can choose to place these two features in a basic 1D polynomial, just a linear combination, or a higher order 2D polynomial. The distinction comes in proper fitting. If the order of this polynomial goes much higher than 2, then we risk over fitting. But we also risk under fitting by only taking the simple linear combination of two features.)

Parameters and a bias are intialized to zero (giving three parameters in the case of a bias, and two features in linear combination.)

Logistic regression is based on the formation of a hypothesis, the calculation of a loss function which is a function of how far away that prediction is from the known label, and ultimately a cost function which is what is minimized. This cost function is a function of the loss and is just a sum of the loss functions over all the samples of a feature. The cost in this case in therefore a single value which can be minimized if the parameters (or weights) of each feature are found such that their combination most closely matches the known label (so that the hypothesis for a sample is close to its label and the loss function will be small.)

Gradient descent (GD) is performed while the error (a difference between successive cost functions) remains above a desired thresholh value. Here that is set a $10^{-6}$, but initialized as a very large value. GD takes those parameters, and performs the above described sequence of calculations and determines slightly alters the parameter values in the hopes of minimizing the cost in each iteration. With successful GD, the error should coverge down to the threshold and the cost function should also converge down to some value.

\begin{figure}[h!]
  \centering
  \includegraphics[width=0.9\linewidth]{./plots/6_classificationperformance.png}
  \caption{\textbf{6}: The cost function converges down to a minimum after a couple hundred iterations of gradient descent.}
  \label{fig:classificationperformance}
\end{figure}

The success of the training (to find the proper weights for each samples' features) can be determined via the activation function, which is the sigmoid in this case. Applying the sigmoid function to the dot product of the weights and features (accounting for the bias) gives the result of what the training algorithm determine to be a short or long GRB (by outputting a value between 0 and 1, whereas the labels are 0 or 1.) Further, a test such as the F1-score quantifies how many true or false identifications the classifier makes given the known labels for each GRB sample.

In this case, the training seems to be failing since it is accepting every single sample as a fast GRB. The F1 score test shows that there are zero true negatives, whereas it is known that there are 185 long GRBs and 50 short GRBs.

And a decision boundary can be constructed based on the line formed by the linear combination of features. It is the boundary where the hypothesis is equal to 0.5, such that anything higher than that up to 1 is a positive idenfication and anything lower is a negative. When plotted against eachother, this linear combination is just the equation for a line which has an ''x-intercept'', a ''y-intercept'', and a slope which can be solved for and plotted.

\begin{figure}[h!]
  \centering
  \includegraphics[width=0.9\linewidth]{./plots/6_decisionboundary.png}
  \caption{\textbf{6}: This boundary dosen't actually seem to be correct. Whereas the F1 score saw that all of the identifications are true positives, the decision boundary does not agree with this graphically. It still shows that there are both types of GRBs on either size of the decision boundary. This could be because my approach of solving the equation of a line is not true in logistic regression where you apply the activation function to the feature combination.}
  \label{fig:decisionboundary}
\end{figure}
\end{section}


\FloatBarrier
\begin{section}{Functions Module (Exercise 7)}
% MY FUNCTIONS

Here are the functions which I call throughout the subquestions to Question 7.

\lstinputlisting{ex7functions.py}

\end{section}

\begin{section}{Exercise 7}
The script used to generate the results is given by:
  
\lstinputlisting{ex7.py}

The result of the script is given by:

\lstinputlisting{ex7output.txt}


With a root node composed of children, which are each composed of children down to the leaf node (containing ideally just one coordinate point), the Barnes Hut Algorithm can be appliedto find the center of mass of sections of space. Here, only a minimum of 12 coordinates per node is required. This can be achieved in 2 manual (as opposed to recursive) iterations of the quadtree building.


\begin{figure}[h!]
  \centering
  \includegraphics[width=0.9\linewidth]{./plots/7_150particles1stiter.png}
  \caption{\textbf{7}: The Barnes Hut Quadtree with initial nodes denoted by color in each corner. The particle of interest has been starred.}
  \label{fig:spatialdensitygrowtht0}
\end{figure}


In the B.H. algorithm, groups of points are succesively merged via calculating their multipole moment. The n=0 multipole moment is just the point mass approximate:

\begin{equation}
  M_0=\sum_i m_a
\end{equation}

Since all these particle masses are equal to 0.0125, the zeroeth multiple moments will only differ by a multiple of that mass. This also means that you get back just the total mass of all the particles upon adding up all the nodes. But the node moments won't be equal as they have different amounts of particles in them (150 in this case).


\begin{figure}[h!]
  \centering
  \includegraphics[width=0.9\linewidth]{./plots/7_150particles2nditer.png}
  \caption{\textbf{7}: The Barnes Hut Quadtree with leaf nodes colored. These leaf nodes contain 12 particles or less and are subsets of the nodes above. The zeroeth multipole moment is found in each of the colored leafnodes (by summing up the mass.) Then the multipole moment of the parent node, and eventually root node can be found in the same way.}
\end{figure}
\end{section}
  



\end{document}
